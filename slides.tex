\documentclass[xcolor=svgnames,11pt]{beamer}

\usepackage{graphicx}

\usepackage[italian, english]{babel} 
\usepackage[utf8x]{inputenc}
\usepackage[gen]{eurosym}

\usepackage{hyperref}
\usepackage{tabularx}
\usepackage{url}

\usetheme{Goettingen}
\definecolor{raspi}{HTML}{bb1142}
\definecolor{green_raspi}{HTML}{75aa28}
\definecolor{myblack}{RGB}{34,34,34}

\usecolortheme[named=green_raspi]{structure}

\beamertemplateshadingbackground{white!15}{raspi!5}
\setbeamertemplate{sidebar canvas right}[vertical shading]%
[top=raspi!50,bottom=white!90]

\setbeamertemplate{blocks}[rounded][shadow=false]
\setbeamercolor{block body}{bg=green_raspi!30}
\setbeamercolor{block title}{bg=green_raspi!90}

\setbeamercovered{transparent}

\title{\textbf{Raspberry Pi - Il computer che hai sempre voluto avere}}
\author{Nicola Corti}
\institute{Linux Day 2014 - Pisa \\ \medskip \includegraphics[height=1.5cm]{ld_logo.png} \hspace{1cm} \includegraphics[height=1.5cm]{gulp_logo.png}}
\logo{gulp_logo.png}
\date{26 ottobre 2014}

 
\begin{document}

\begin{frame}
	\titlepage
\end{frame}

\section{Introduzione}

\begin{frame}{Cosa \`e il Raspberry Pi}
\begin{center}
\includegraphics[width=9cm]{isapc.png}
\end{center}
\end{frame}

\begin{frame}{Cosa \`e il Raspberry Pi}

\pause

Il Raspberry Pi \`e a tutti gli effetti un \textbf{computer}, che ci permette sostanzialmente di effettuare le stesse operazioni che faremo con un computer classico.

\pause

\begin{block}{Cosa \`e?}
\emph{
Il Raspberry Pi è un single-board computer (un calcolatore implementato su una sola scheda elettronica).}
\begin{flushright}
\begin{footnotesize}
\emph{Da Wikipedia, l'enciclopedia libera.}
\end{footnotesize}
\end{flushright}

\end{block}

\pause
\medskip

...per\`o \`e pi\`u \textbf{piccolo}!

\end{frame}

\begin{frame}{Cosa \`e il Raspberry Pi}
\begin{center}
\includegraphics[width=9cm]{raspi.png}
\end{center}
\end{frame}


\begin{frame}{Come nasce il Raspberry Pi}
\begin{center}
\includegraphics[width=6cm]{logo_raspi.png}
\end{center}
\pause

Il Raspberry Pi nasce nel Regno Unito, realizzato dalla \emph{Raspberry Pi Foundation}

\begin{center}
\url{http://www.raspberrypi.org/}
\end{center}

\pause
\medskip

\`E nato con l'intendo di creare un computer:
\pause
\begin{itemize}
\item Per avvicinare alla programmazione,
\pause
\item Per la didattica nelle scuole,
\pause
\item Che sia economicamente accessibile.
\end{itemize}
\end{frame}

\begin{frame}{Raspberry Pi - Model B}
\begin{center}
\includegraphics[width=9cm]{scheme.png}
\end{center}
\end{frame}

\begin{frame}{Quali modelli di Raspberry Pi}
\begin{center}
\includegraphics[width=9.5cm]{table.png}
\end{center}
\end{frame}

\begin{frame}{Cosa serve per far funzionare un Raspberry Pi}
Per iniziare a divertirci con il nostro Raspberry Pi avremo bisogno di:
\pause
\begin{description}
\item[Alimentatore] Micro USB, Output a 1200 $mA$.
\pause
\item[Scheda SD] Da almeno 2 GB, meglio se da 4 GB (e possibilmente di classe 10).
\pause
\item[Rete] Connessione ethernet ad internet.
\pause
\item[Input] Mouse e tastiera USB (consigliati).
\pause
\item[Monitor] Con interfaccia HDMI o DVI (consigliati), oppure un televisore con entrata RCA.
\end{description}
\end{frame}

\begin{frame}[fragile]{Accessori per il Raspberry Pi}
Estendiamo il nostro Raspberry Pi tramite:
\vspace{1cm}

\begin{columns}
    \begin{column}{0.5\textwidth}
	\begin{itemize}
	\item Pi-Camera
	\pause
	\item Touch Screen LCD
	\pause
	\item Hub USB
	\pause
	\item Dongle Wifi (Bluetooth o 3G)
	\pause
	\item Case
	\end{itemize}
    \end{column}
    \begin{column}{0.5\textwidth}
    \begin{center}
      \includegraphics<1>[height=4cm]{picamera.png}
      \includegraphics<2>[height=4cm]{pimonitor.png}
      \includegraphics<3>[height=4cm]{pihub.png}
      \includegraphics<4>[height=4cm]{piwifi.png}
      \includegraphics<5>[height=4cm]{picase.png}            
    \end{center}
    \end{column}
  \end{columns}
\end{frame}

\begin{frame}{Dove comprare il Raspberry Pi}

\begin{center}
\includegraphics[width=5cm]{wherebuy.png}
\end{center}

\`E possibile acquistare il Raspberry Pi presso uno dei distributori ufficiali, oppure anche su qualsiasi altro shop online che venda articoli di elettronica.

\medskip

Il costo per i modelli B/B+ si aggira intorno ai 35 euro.

\end{frame}

\section{Use cases}

\begin{frame}{Media Center}
\end{frame}

\begin{frame}{Server Domestico}
\end{frame}

\begin{frame}{Torrent Server}
\end{frame}

\begin{frame}{Console Arcade}
\end{frame}

\begin{frame}{Terminal Server}
\end{frame}

\begin{frame}{Home automation/security}
\end{frame}

\begin{frame}{Learn to programming}
\end{frame}

\begin{frame}{Voip Server}
\end{frame}

\section{Distribuzioni Linux}

\begin{frame}{Raspbian}
\end{frame}

\begin{frame}{Raspbmc}
\end{frame}

\begin{frame}{NOOBS}
\end{frame}

\section{Prima installazione}

\begin{frame}{1) Scaricare NOOBS}
\end{frame}

\begin{frame}{2) Formattare la scheda SD}
\end{frame}

\begin{frame}{3) Copiare NOOBS su scheda SD} 
\end{frame}

\begin{frame}{4) Avviare il Raspberry Pi}
\end{frame}

\begin{frame}{5) Scegliere i S.O.}
\end{frame}

\begin{frame}{6) Attendere...}
\end{frame}

\end{document}
