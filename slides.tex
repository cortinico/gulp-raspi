\documentclass[xcolor=svgnames,11pt]{beamer}

\usepackage{graphicx}

\usepackage[italian, english]{babel} 
\usepackage[utf8x]{inputenc}

\usepackage{hyperref}
\usepackage{listings}

\usepackage{tabularx}
%\usepackage{color}
\usepackage{url}

\usetheme{Goettingen}
\definecolor{android}{HTML}{bb1142}
\definecolor{myblack}{RGB}{34,34,34}

\usecolortheme[named=myblack]{structure}

\beamertemplateshadingbackground{white!15}{android!5}
\setbeamertemplate{sidebar canvas right}[vertical shading]%
[top=android!50,bottom=white!90]

\setbeamertemplate{blocks}[rounded][shadow=false]
\setbeamercolor{block body}{bg=android!40}
\setbeamercolor{block title}{bg=android!90}


\setbeamercovered{transparent}


\title{\textbf{Raspberry Pi - Il computer che hai sempre voluto avere}}
\author{Nicola Corti}
\institute{Linux Day 2014 - Pisa \\ \medskip \includegraphics[height=1.5cm]{ld_logo.png} \hspace{1cm} \includegraphics[height=1.5cm]{gulp_logo.png}}
\logo{gulp_logo.png}
\date{26 ottobre 2014}

 
\begin{document}

\begin{frame}
	\titlepage
\end{frame}

\section{Introduzione}

\begin{frame}{Cosa \`e il Raspberry Pi}
\end{frame}

\begin{frame}{Come nasce il Raspberry Pi}
\end{frame}

\begin{frame}{Quali modelli di Raspberry Pi}
\end{frame}

\begin{frame}{Cosa serve per far funzionare un Raspberry Pi}
\end{frame}

\begin{frame}{Accessori per il Raspberry Pi}
\end{frame}

\begin{frame}{Dove comprare il Raspberry Pi}
\end{frame}

\section{Use cases}

\begin{frame}{Media Center}
\end{frame}

\begin{frame}{Server Domestico}
\end{frame}

\begin{frame}{Torrent Server}
\end{frame}

\begin{frame}{Console Arcade}
\end{frame}

\begin{frame}{Terminal Server}
\end{frame}

\begin{frame}{Home automation/security}
\end{frame}

\begin{frame}{Learn to programming}
\end{frame}

\begin{frame}{Voip Server}
\end{frame}

\section{Distribuzioni Linux}

\begin{frame}{Raspbian}
\end{frame}

\begin{frame}{Pidora}
\end{frame}

\begin{frame}{Arch}
\end{frame}

\begin{frame}{Raspbmc}
\end{frame}

\begin{frame}{NOOBS}
\end{frame}

\section{Prima installazione}

\begin{frame}{1) Scaricare NOOBS}
\end{frame}

\begin{frame}{2) Formattare la scheda SD}
\end{frame}

\begin{frame}{3) Copiare NOOBS su scheda SD} 
\end{frame}

\begin{frame}{4) Avviare il Raspberry Pi}
\end{frame}

\begin{frame}{5) Scegliere i S.O.}
\end{frame}

\begin{frame}{6) Attendere...}
\end{frame}

\end{document}
